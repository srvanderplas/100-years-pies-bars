% Options for packages loaded elsewhere
\PassOptionsToPackage{unicode}{hyperref}
\PassOptionsToPackage{hyphens}{url}
\PassOptionsToPackage{dvipsnames,svgnames,x11names}{xcolor}
%
\documentclass[
  10pt,
  letterpaper,
  twocolumn]{article}

\usepackage{amsmath,amssymb}
\usepackage{iftex}
\ifPDFTeX
  \usepackage[T1]{fontenc}
  \usepackage[utf8]{inputenc}
  \usepackage{textcomp} % provide euro and other symbols
\else % if luatex or xetex
  \usepackage{unicode-math}
  \defaultfontfeatures{Scale=MatchLowercase}
  \defaultfontfeatures[\rmfamily]{Ligatures=TeX,Scale=1}
\fi
\usepackage{lmodern}
\ifPDFTeX\else  
    % xetex/luatex font selection
  \setmainfont[]{Times New Roman}
\fi
% Use upquote if available, for straight quotes in verbatim environments
\IfFileExists{upquote.sty}{\usepackage{upquote}}{}
\IfFileExists{microtype.sty}{% use microtype if available
  \usepackage[]{microtype}
  \UseMicrotypeSet[protrusion]{basicmath} % disable protrusion for tt fonts
}{}
\makeatletter
\@ifundefined{KOMAClassName}{% if non-KOMA class
  \IfFileExists{parskip.sty}{%
    \usepackage{parskip}
  }{% else
    \setlength{\parindent}{0pt}
    \setlength{\parskip}{6pt plus 2pt minus 1pt}}
}{% if KOMA class
  \KOMAoptions{parskip=half}}
\makeatother
\usepackage{xcolor}
\setlength{\emergencystretch}{3em} % prevent overfull lines
\setcounter{secnumdepth}{-\maxdimen} % remove section numbering
% Make \paragraph and \subparagraph free-standing
\ifx\paragraph\undefined\else
  \let\oldparagraph\paragraph
  \renewcommand{\paragraph}[1]{\oldparagraph{#1}\mbox{}}
\fi
\ifx\subparagraph\undefined\else
  \let\oldsubparagraph\subparagraph
  \renewcommand{\subparagraph}[1]{\oldsubparagraph{#1}\mbox{}}
\fi


\providecommand{\tightlist}{%
  \setlength{\itemsep}{0pt}\setlength{\parskip}{0pt}}\usepackage{longtable,booktabs,array}
\usepackage{calc} % for calculating minipage widths
% Correct order of tables after \paragraph or \subparagraph
\usepackage{etoolbox}
\makeatletter
\patchcmd\longtable{\par}{\if@noskipsec\mbox{}\fi\par}{}{}
\makeatother
% Allow footnotes in longtable head/foot
\IfFileExists{footnotehyper.sty}{\usepackage{footnotehyper}}{\usepackage{footnote}}
\makesavenoteenv{longtable}
\usepackage{graphicx}
\makeatletter
\def\maxwidth{\ifdim\Gin@nat@width>\linewidth\linewidth\else\Gin@nat@width\fi}
\def\maxheight{\ifdim\Gin@nat@height>\textheight\textheight\else\Gin@nat@height\fi}
\makeatother
% Scale images if necessary, so that they will not overflow the page
% margins by default, and it is still possible to overwrite the defaults
% using explicit options in \includegraphics[width, height, ...]{}
\setkeys{Gin}{width=\maxwidth,height=\maxheight,keepaspectratio}
% Set default figure placement to htbp
\makeatletter
\def\fps@figure{htbp}
\makeatother
\newlength{\cslhangindent}
\setlength{\cslhangindent}{1.5em}
\newlength{\csllabelwidth}
\setlength{\csllabelwidth}{3em}
\newlength{\cslentryspacingunit} % times entry-spacing
\setlength{\cslentryspacingunit}{\parskip}
\newenvironment{CSLReferences}[2] % #1 hanging-ident, #2 entry spacing
 {% don't indent paragraphs
  \setlength{\parindent}{0pt}
  % turn on hanging indent if param 1 is 1
  \ifodd #1
  \let\oldpar\par
  \def\par{\hangindent=\cslhangindent\oldpar}
  \fi
  % set entry spacing
  \setlength{\parskip}{#2\cslentryspacingunit}
 }%
 {}
\usepackage{calc}
\newcommand{\CSLBlock}[1]{#1\hfill\break}
\newcommand{\CSLLeftMargin}[1]{\parbox[t]{\csllabelwidth}{#1}}
\newcommand{\CSLRightInline}[1]{\parbox[t]{\linewidth - \csllabelwidth}{#1}\break}
\newcommand{\CSLIndent}[1]{\hspace{\cslhangindent}#1}

\usepackage{sdss2020} % Uses Times Roman font (either newtx or times package)
\usepackage{url}
\usepackage{hyperref}
\usepackage{latexsym}
\usepackage{amsmath, amsthm, amsfonts}
\usepackage{algorithm, algorithmic}
\usepackage[dvipsnames]{xcolor} % colors
\newcommand{\mt}[1]{{\textcolor{blue}{#1}}}
\newcommand{\svp}[1]{{\textcolor{RedOrange}{#1}}}
\makeatletter
\makeatother
\makeatletter
\makeatother
\makeatletter
\@ifpackageloaded{caption}{}{\usepackage{caption}}
\AtBeginDocument{%
\ifdefined\contentsname
  \renewcommand*\contentsname{Table of contents}
\else
  \newcommand\contentsname{Table of contents}
\fi
\ifdefined\listfigurename
  \renewcommand*\listfigurename{List of Figures}
\else
  \newcommand\listfigurename{List of Figures}
\fi
\ifdefined\listtablename
  \renewcommand*\listtablename{List of Tables}
\else
  \newcommand\listtablename{List of Tables}
\fi
\ifdefined\figurename
  \renewcommand*\figurename{Figure}
\else
  \newcommand\figurename{Figure}
\fi
\ifdefined\tablename
  \renewcommand*\tablename{Table}
\else
  \newcommand\tablename{Table}
\fi
}
\@ifpackageloaded{float}{}{\usepackage{float}}
\floatstyle{ruled}
\@ifundefined{c@chapter}{\newfloat{codelisting}{h}{lop}}{\newfloat{codelisting}{h}{lop}[chapter]}
\floatname{codelisting}{Listing}
\newcommand*\listoflistings{\listof{codelisting}{List of Listings}}
\makeatother
\makeatletter
\@ifpackageloaded{caption}{}{\usepackage{caption}}
\@ifpackageloaded{subcaption}{}{\usepackage{subcaption}}
\makeatother
\makeatletter
\@ifpackageloaded{tcolorbox}{}{\usepackage[skins,breakable]{tcolorbox}}
\makeatother
\makeatletter
\@ifundefined{shadecolor}{\definecolor{shadecolor}{rgb}{.97, .97, .97}}
\makeatother
\makeatletter
\makeatother
\makeatletter
\makeatother
\ifLuaTeX
  \usepackage{selnolig}  % disable illegal ligatures
\fi
\IfFileExists{bookmark.sty}{\usepackage{bookmark}}{\usepackage{hyperref}}
\IfFileExists{xurl.sty}{\usepackage{xurl}}{} % add URL line breaks if available
\urlstyle{same} % disable monospaced font for URLs
\hypersetup{
  pdftitle={100 years of Pies vs.~Bars},
  pdfauthor={Maksuda Atkar Toma; Susan Vanderplas},
  colorlinks=true,
  linkcolor={blue},
  filecolor={Maroon},
  citecolor={Blue},
  urlcolor={Blue},
  pdfcreator={LaTeX via pandoc}}

\title{100 years of Pies vs.~Bars}
\author{
Maksuda Atkar Toma\\
Statistics Department, \\University of Nebraska, Lincoln\\
{\tt \href{mailto:mtoma2@huskers.unl.edu}{mtoma2@huskers.unl.edu}}\\
\\\And
Susan Vanderplas\\
Statistics Department, \\University of Nebraska, Lincoln\\
{\tt \href{mailto:svanderplas2@unl.edu}{svanderplas2@unl.edu}}\\
}
\date{}

\begin{document}
\maketitle
\ifdefined\Shaded\renewenvironment{Shaded}{\begin{tcolorbox}[enhanced, sharp corners, interior hidden, boxrule=0pt, borderline west={3pt}{0pt}{shadecolor}, frame hidden, breakable]}{\end{tcolorbox}}\fi

\hypertarget{historical-background}{%
\section{Historical Background}\label{historical-background}}

William Playfair introduced the pie and bar charts in the early
1800s\svp{, and by the turn of the next century, strong opinions had
been publicized in textbooks and manuals, and statisticians began
conducting experimental studies comparing the two.} \svp{These studies
have contradictory results:} Huhn (1927) concludes that bar charts are
much more useful and efficient in the majority of situations, even
though pie charts have specific applications\svp{, while Croxton and
Stryker (1927) found that pie charts may match or surpass them in
certain situations.} \svp{Much later, Cleveland and McGill (1984)
conducted experiments on simple graphical elements, stating that aligned
length judgments are more accurate than judgments based on area or
angle, with the implication that bar charts are more accurately
perceived than pie charts.} \svp{Their hierarchy of feature
comprehension is often assumed to be experimentally derived, but the
experiments presented in the papers (Cleveland and McGill 1984, 1985,
1986) only covered a few elements of this hierarchy.}

\svp{Here, we re-examine the historical literature surrounding the use
of pie and bar charts, examining recommendations about the use of pie
and bar charts, tracing these guidelines back to empirical studies, and
evaluating whether the recommendations are well-supported by the
studies.} \svp{We begin with an initial set of four studies from the
late 1920s and early 1930s: Eells (1926), Croxton and Stryker (1927),
Huhn (1927), and Croxton and Stein (1932).} \svp{We trace these studies
forward in time, examining the claims made about the studies as well as
the recommendations made in citing papers.} \svp{In addition, we lay out
the different experimental designs used in each experimental study, with
the goal of assessing how the design of each study might contribute to
the ultimate results.} \mt{They demonstrated that the choice of
visualization depends on the task and the data structure. Bars are
preferable for comparisons and time-series data, while circles may be
suitable for single distributions with grouped components.} \mt{Eells
(1926, 1927) found that pie charts (circles) were more accurate than bar
charts when presenting simple distributions, such as 50-50 or 25-75
relationships. However, bar charts were superior for more complex
distributions and comparative tasks due to their linear scale, which
facilitates comparisons across multiple diagrams.} \mt{Vohn (1927)
argued that bar charts offer advantages for tasks requiring comparisons
across time series or multiple distributions, owing to their standard
scale and ease of labeling, while circles are more suited for showing
numerous components grouped into categories. Frederic and Croxton (1932)
compared bars, squares, circles, and cubes for simple comparisons. They
confirmed that bar charts yielded the most accurate estimates,
outperforming circles, squares, and cubes. Squares and circles were
comparable in accuracy, while cubes consistently performed the worst due
to the challenge of representing three-dimensional volume in
two-dimensional drawings. These findings emphasize that linear encodings
(e.g., bars) are more effective than area- or volume-based encodings for
precise quantitative judgments.} \svp{Asking for absolute estimates of
differences (A - B) likely gives the advantage to bar charts, while
asking for part-to-whole estimates (A/(A+B)) would be expected to give
the advantage to pie charts; asking for A/B ratio comparisons may
produce yet a different result.}

\mt{In late 70's and 80's Studies on graphical perception reveal that
position is the most accurate visual encoding method, followed by length
and angle, with tasks involving area, volume, curvature, shading, and
color saturation needing more reliability. Baird (1970) demonstrated
that length is perceived more accurately than area and volume, which are
often underestimated. Baird and Noma (1978) highlighted how visual cues,
such as frames, improve accuracy by providing reference structures.
Experiments comparing position and length judgments found that position
tasks were significantly more accurate, with length judgments resulting
in 40--250\% higher errors. Similarly, comparisons of position and angle
judgments revealed that pie charts, which rely on angle encodings, are
far less effective than bar charts, with position judgments being nearly
twice as accurate and accounting for significantly fewer errors. Biases
and outliers were common, with position accuracy declining as distances
increased. These findings emphasize the need to avoid traditional
visualizations like pie and divided bar charts, recommending
alternatives such as dot charts or framed rectangular charts, which
leverage position-based encodings to improve accuracy and comprehension.
These insights underscore the importance of redesigning data
visualizations to optimize perceptual accuracy and usability.}

\svp{While these studies are the earliest systematic studies of
statistical graphics we could locate, modern statistical graphics
research might be said to date from Cleveland \& McGill's work on the
perception of simple graphical elements; to include this work as well,
we also look forward and backward from these papers, examining both
cited papers and papers describing their work in subsequent research.}

\hypertarget{purpose-and-motivation}{%
\section{Purpose and Motivation}\label{purpose-and-motivation}}

Visualizing quantitative data effectively is essential in
decision-making, scientific research, and education. Graphical tools
must convey data precisely while engaging users to ensure optimal
understanding. Yet, disagreements persist about which chart types best
balance clarity, accessibility, and accuracy. Motivated by the need for
improved visual communication, this research synthesizes findings from
decades of studies in graphical perception, including experimental
evaluations and historical insights. It aims to explore the relative
merits of widely used visualizations, including bar charts, pie charts,
and hybrid formats like grables, (Hink, Eustace, and S. Wogalter 1998)
which combine numerical tables with graphical clarity. The goal is to
establish task-specific design principles and empower practitioners and
educators to build more effective visual tools. (\textbf{Check this
paragraph, took help from AI})

We will try to find out the importance of understanding the historical
context of graphical methods, highlighting how their development
reflects broader trends in statistical communication. We want to explore
how modern visualization software (e.g., Tableau, Power BI) aligns with
Cleveland and McGill's perceptual task hierarchy. Assess whether newer
tools mitigate the limitations of suboptimal encodings like area and
angle.\\
Another idea could be to\,investigate whether dynamic visualizations
(e.g., interactive scatterplots, time-lapse animations) improve
perceptual accuracy for traditionally low-ranking encodings such as area
and slope.

\hypertarget{problem}{%
\section{Problem}\label{problem}}

While charts like bar and pie graphs are staples of data communication,
their utility varies widely depending on the context. Pie charts often
appeal due to their simplicity and visual aesthetics but falter in
scenarios requiring detailed quantitative comparisons due to their
reliance on angular and area-based judgments, which humans inherently
struggle to process accurately. Bar charts, on the other hand, excel in
tasks requiring precision, as they leverage a common positional scale.
However, they lack the visual immediacy that non-technical audiences may
seek. Meanwhile, emerging formats like grables combine the precision of
tables with graphical representation but are underexplored in practice.
This research addresses these gaps, seeking to clarify how these
visualizations perform across different cognitive and perceptual tasks.
In this modern era where AI has become very available to make such kinds
of charts within a minute, how do humans' perspectives change in
different fields? \mt{\textbf{Paraphrased this section}}

\hypertarget{methods}{%
\section{Methods}\label{methods}}

This study draws upon seminal experiments, including Cleveland and
McGill's hierarchy of graphical encodings and Eells' foundational
comparisons of pie and bar charts. A detailed review of tasks---value
retrieval, trend analysis, and comparisons---provides insights into the
cognitive load and accuracy associated with each visualization type.
Historical analysis traces the evolution of statistical graphics, from
William Playfair's pioneering work in bar and pie charts to modern
hybrid visualizations. Experimental findings are organized into a
taxonomy ranking encodings like position, length, area, and color in
terms of effectiveness and alignment with specific tasks.

\hypertarget{conclusion}{%
\section{Conclusion}\label{conclusion}}

This research highlights the importance of tailoring visualizations to
specific tasks and user contexts. Bar charts, emphasizing positional
accuracy, remain the gold standard for detailed analyses, while pie
charts can serve as accessible tools for summarizing information.
Grables represent a promising innovation, combining clarity with
precision for complex data sets. By integrating historical perspectives
with modern insights from graphical perception studies, this research
provides a comprehensive roadmap for designing effective visual tools.
Practitioners, educators, and researchers are encouraged to adopt these
principles, ensuring their visualizations communicate data effectively
and foster deeper engagement and understanding.

This study emphasizes the relevance of designing visualizations for
specific tasks and user settings. Bar charts, which emphasize positional
accuracy, remain the gold standard for extensive analysis, although pie
charts can be useful tools for summarizing information. Grables are a
potential innovation that combines clarity and precision in big data
sets. This study provides a complete roadmap for developing effective
visual aids by combining historical viewpoints with contemporary
insights from graphical perception studies.

\mt{This article also investigates the relevance of human visual
perception in current tools such as Tableau and PowerBI Practitioners,
educators, and researchers are advised to follow these guidelines to
ensure that their visualizations effectively communicate data while
fostering deeper engagement and understanding.}

\hypertarget{refs}{}
\begin{CSLReferences}{1}{0}
\leavevmode\vadjust pre{\hypertarget{ref-clevelandGraphicalPerceptionTheory1984}{}}%
Cleveland, William S., and Robert McGill. 1984. {``Graphical Perception:
Theory, Experimentation, and Application to the Development of Graphical
Methods.''} \emph{Journal of the American Statistical Association} 79
(387): 531--54. \url{https://doi.org/10.1080/01621459.1984.10478080}.

\leavevmode\vadjust pre{\hypertarget{ref-clevelandGraphicalPerceptionGraphical1985}{}}%
---------. 1985. {``Graphical Perception and Graphical Methods for
Analyzing Scientific Data.''} \emph{Science} 229 (4716): 828--33.
\url{https://doi.org/10.1126/science.229.4716.828}.

\leavevmode\vadjust pre{\hypertarget{ref-clevelandExperimentGraphicalPerception1986}{}}%
---------. 1986. {``An Experiment in Graphical Perception.''}
\emph{International Journal of Man-Machine Studies} 25 (5): 491--500.
\url{https://doi.org/10.1016/S0020-7373(86)80019-0}.

\leavevmode\vadjust pre{\hypertarget{ref-croxtonGraphicComparisonsBars1932}{}}%
Croxton, Frederick E., and Harold Stein. 1932. {``Graphic Comparisons by
Bars, Squares, Circles, and Cubes.''} \emph{Journal of the American
Statistical Association} 27 (177): 54--60.

\leavevmode\vadjust pre{\hypertarget{ref-croxtonBarChartsCircle1927}{}}%
Croxton, Frederick E., and Roy E. Stryker. 1927. {``Bar Charts Versus
Circle Diagrams.''} \emph{Journal of the American Statistical
Association} 22 (160): 473--82. \url{https://doi.org/10.2307/2276829}.

\leavevmode\vadjust pre{\hypertarget{ref-eellsRelativeMeritsCircles1926}{}}%
Eells, Walter Crosby. 1926. {``The Relative Merits of Circles and Bars
for Representing Component Parts.''} \emph{Journal of the American
Statistical Association} 21 (154): 119--32.
\url{https://doi.org/10.2307/2277140}.

\leavevmode\vadjust pre{\hypertarget{ref-hinkGrablesEnableExtraction1998}{}}%
Hink, Jessica K, Jason K Eustace, and Michael S. Wogalter. 1998. {``Do
Grables Enable the Extraction of Quantitative Information Better Than
Pure Graphs or Tables?''} \emph{International Journal of Industrial
Ergonomics} 22 (6): 439--47.
\url{https://doi.org/10.1016/S0169-8141(97)00017-6}.

\leavevmode\vadjust pre{\hypertarget{ref-vonhuhnFurtherStudiesGraphic1927}{}}%
Huhn, R. von. 1927. {``Further Studies in the Graphic Use of Circles and
Bars: A Discussion of the Eell's Experiment.''} \emph{Journal of the
American Statistical Association} 22 (157): 31--39.
\url{https://doi.org/10.2307/2277346}.

\end{CSLReferences}


\bibliographystyle{sdss2020} % Please do not change the bibliography style

\end{document}
